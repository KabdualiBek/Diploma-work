\documentclass{article}
\usepackage[14pt]{extsizes}
\usepackage[utf8]{inputenc}
\usepackage[russianb]{babel}
\usepackage{vmargin}
\usepackage{setspace}
\usepackage{amsmath}
\usepackage{amsfonts}
\usepackage{amssymb}
\usepackage{amsfonts}
\usepackage{amssymb}\setpapersize{A4}
\setmarginsrb{2.5cm}{1.5cm}{1.5cm}{1.5cm}{0pt}{0mm}{0pt}{13mm}
\usepackage{indentfirst}
\usepackage{amsmath}
\usepackage{amssymb}
\usepackage{amsfonts}
\usepackage{amsthm}
\usepackage{graphicx}
\usepackage{float}
\usepackage{subfig}
\usepackage{listings}
\graphicspath{{.}}
\newcommand*{\hm}[1]{#1\nobreak\discretionary{}%
	{\hbox{$\mathsurround=0pt #1$}}{}}
\renewcommand{\le}{\leqslant}
\renewcommand{\ge}{\geqslant}
\newcommand{\h}{\textsf}
\setcounter{secnumdepth}{2}

\newtheorem{theorem}{Теорема}
\newtheorem{lemma}{Лемма}
\newtheorem{statement}{Утверждение}
\theoremstyle{definition}
\newtheorem{definition}{Определение}
\newtheorem*{remark}{Замечание}

\usepackage{listings}
\usepackage{xcolor}

\definecolor{codegreen}{rgb}{0,0.6,0}
\definecolor{codegray}{rgb}{0.5,0.5,0.5}
\definecolor{codepurple}{rgb}{0.58,0,0.82}
\definecolor{backcolour}{rgb}{0.95,0.95,0.92}

\lstdefinestyle{mystyle}{
	backgroundcolor=\color{backcolour},   
	commentstyle=\color{codegreen},
	keywordstyle=\color{magenta},
	numberstyle=\scriptsize\color{black},
	stringstyle=\color{codepurple},
	basicstyle=\ttfamily\footnotesize,
	breakatwhitespace=false,         
	breaklines=true,                 
	captionpos=b,    
	extendedchars=true,                
	keepspaces=true,                 
	numbers=left,                    
	numbersep=5pt,                  
	showspaces=false,                
	showstringspaces=false,
	showtabs=false,                  
	tabsize=2
}

\lstset{style=mystyle, 
	literate={а}{{\selectfont\char224}}1
	{б}{{\selectfont\char225}}1
	{в}{{\selectfont\char226}}1
	{г}{{\selectfont\char227}}1
	{д}{{\selectfont\char228}}1
	{е}{{\selectfont\char229}}1
	{ё}{{\"e}}1
	{ж}{{\selectfont\char230}}1
	{з}{{\selectfont\char231}}1
	{и}{{\selectfont\char232}}1
	{й}{{\selectfont\char233}}1
	{к}{{\selectfont\char234}}1
	{л}{{\selectfont\char235}}1
	{м}{{\selectfont\char236}}1
	{н}{{\selectfont\char237}}1
	{о}{{\selectfont\char238}}1
	{п}{{\selectfont\char239}}1
	{р}{{\selectfont\char240}}1
	{с}{{\selectfont\char241}}1
	{т}{{\selectfont\char242}}1
	{у}{{\selectfont\char243}}1
	{ф}{{\selectfont\char244}}1
	{х}{{\selectfont\char245}}1
	{ц}{{\selectfont\char246}}1
	{ч}{{\selectfont\char247}}1
	{ш}{{\selectfont\char248}}1
	{щ}{{\selectfont\char249}}1
	{ъ}{{\selectfont\char250}}1
	{ы}{{\selectfont\char251}}1
	{ь}{{\selectfont\char252}}1
	{э}{{\selectfont\char253}}1
	{ю}{{\selectfont\char254}}1
	{я}{{\selectfont\char255}}1
	{А}{{\selectfont\char192}}1
	{Б}{{\selectfont\char193}}1
	{В}{{\selectfont\char194}}1
	{Г}{{\selectfont\char195}}1
	{Д}{{\selectfont\char196}}1
	{Е}{{\selectfont\char197}}1
	{Ё}{{\"E}}1
	{Ж}{{\selectfont\char198}}1
	{З}{{\selectfont\char199}}1
	{И}{{\selectfont\char200}}1
	{Й}{{\selectfont\char201}}1
	{К}{{\selectfont\char202}}1
	{Л}{{\selectfont\char203}}1
	{М}{{\selectfont\char204}}1
	{Н}{{\selectfont\char205}}1
	{О}{{\selectfont\char206}}1
	{П}{{\selectfont\char207}}1
	{Р}{{\selectfont\char208}}1
	{С}{{\selectfont\char209}}1
	{Т}{{\selectfont\char210}}1
	{У}{{\selectfont\char211}}1
	{Ф}{{\selectfont\char212}}1
	{Х}{{\selectfont\char213}}1
	{Ц}{{\selectfont\char214}}1
	{Ч}{{\selectfont\char215}}1
	{Ш}{{\selectfont\char216}}1
	{Щ}{{\selectfont\char217}}1
	{Ъ}{{\selectfont\char218}}1
	{Ы}{{\selectfont\char219}}1
	{Ь}{{\selectfont\char220}}1
	{Э}{{\selectfont\char221}}1
	{Ю}{{\selectfont\char222}}1
	{Я}{{\selectfont\char223}}1
}

\begin{document}
	\begin{titlepage}
		\begin{center}
			\includegraphics{msu_logo.jpg}
			\small
			~\\[0.1cm]
			Московский государственный университет имени М.В.Ломоносова
			~\\[0.1cm]
			Факультет вычислительной математики и кибернетики
			~\\[0.1cm]
			Кафедра математической физики
			~\\[1.0cm]
			\normalsize
		\end{center}
	
  	\vspace{1.5cm}
\begin{center}
	\textbf{\large{Кабдуали Бек Ильясулы}}
	\\[2cm]
	\textbf{\large {<<Исследование нелокальной задачи для уравнения переноса при специальных предположениях>>}}
	\\[1.5cm]
	
	\textbf{\large Выпускная квалификационная работа}
	\\[1cm]
	\begin{normalsize}
		\begin{flushright}
			\small
			Научный руководитель:
			\\
			профессор кафедры математической физики,\\ доктор физико-математических наук
			\\
			Тихонов И.\,В.
		\end{flushright}
		\normalsize
	\end{normalsize}
	\vfill 
	
	\small{Москва, 2019}
\end{center} 

	\end{titlepage}

\newpage

\tableofcontents

\newpage

\section*{Введение}
\addcontentsline{toc}{section}{Введение}
Рассматривается нелокальная задача для уравнения простого переноса с поглощением. 
Путём элементарных преобразований получается редуцированная задача,
которая является частным случаем абстрактной нелокальной задачи. 

Далее, следуют определения и утверждения из функционального анализа и теории полугрупп, необходимые для решения абстрактной задачи. После ставится сама абстрактная задача и на основании приведённой выше теории выводится её решение.

Приводятся иллюстрации применения теоретических результатов, полученных при решении абстрактной задачи.

Даётся описание программы, которая строит график решения исходной нелокальной задачи при 
$\sigma(x) \equiv 0$, $\gamma(t) \equiv 0$.

В конце приводятся результаты вычислительных экспериментов.

\newpage

\section{Обзор литературы}
Общая постановка нелокальной задачи и полное исследование
в случае произвольной суперустойчивой полугруппы имеются в \cite{Tikhonov1}.
Теоретический пример, связанный с многомерным уравнением простого переноса,
подробно разобран в \cite{Tikhonov2}.
Суперустойчивые полугруппы подробно изучались в работах \cite{Balakrishnan_1, Balakrishnan_2}.
При подготовке отчёта использовались стандартные понятия из теории полугрупп \cite{Dunford_Schwartz, Pazy},
а также из теории функций и функционального анализа \cite{Kolmogorov_Fomin}, \cite{Trenogin}.
Основные сведения из теории дифференциальных уравнений в частных производных можно найти в \cite{Filippov}.

\newpage

\section{Постановка задачи}

\subsection{Постановка нелокальной задачи}
Пусть некоторая субстанция перемещается вдоль одномерной среды с постоянной скоростью и, быть может, поглащается этой средой.
Такой процесс моделирует уравнение простого переноса
\begin{equation} \label{transport equation}
	u_t + au_x + \sigma(x)u = 0, \qquad 0 \le x \le l, \quad 0 \le t \le T.
\end{equation}
Здесь $x$~--- пространственная координата, $t$~--- время, $u(x,t)$~--- плотность вещества в точке $x$ в момент времени $t$, 
константа $ a > 0 $~--- скорость переноса, функция $ \sigma(x) \ge 0 $ при $ 0 \le x \le l $~--- коэффициент поглощения.

К уравнению \eqref{transport equation} добавляем граничное условие
\begin{equation} \label{boundary condition}
	u(0,t) = \gamma(t), \qquad t \ge 0,
\end{equation}
 и нелокальное усреднение по времени
\begin{equation} \label{integral averaging}
	\int\limits_{0}^{T} \eta(t) u(x, t)\,dt = \psi(x), 	\qquad 0 \le x \le l.
\end{equation}
Начальное условие 
\begin{equation*}
	u(x,0) = u_0(x), \qquad 0 \le x \le l,
\end{equation*}
сейчас не задано.

В совкупности соотношения \eqref{transport equation}--\eqref{integral averaging} образуют задачу
\begin{equation} \label{nonlocal problem}
\begin{cases}
	\; u_t + au_x + \sigma(x)u = 0, \qquad 0 \le x \le l, \quad 0 \le t \le T, \\[3mm]
	\; u(0,t) = \gamma(t), \\[1mm]
	\; \int\limits_{0}^{T} \eta(t) u(x,t)\,dt = \psi(x).
\end{cases}
\end{equation}
Функции $ \gamma(t), \, \eta(t), \, \psi(x)$ заданы.
Коэффициенты $ a, \, \sigma(x)$ тоже заданы. 
Неизвестной являетя функция $ u(x,t) $.

Основное предположение: считаем, что весовая функция $ \eta(t) $ является кусочно постоянной
\begin{equation} \label{eta(t)}
\eta(t) = 
	\begin{cases}
	\; \alpha_1, \qquad \tau_0 \le t < \tau_1, \\
	\; \alpha_2, \qquad \tau_1 < t < \tau_2, \\
	\; .\phantom{-}.\phantom{-}.\phantom{-}.\phantom{-}.\phantom{-}.\phantom{-}.\phantom{-}.\phantom{-}. \\
	\; \alpha_p, \qquad \tau_{p - 1} < t \le \tau_p.
	\end{cases}
\end{equation}
Здесь $\alpha_1$, $\alpha_2$, ..., $\alpha_p$~--- вещественные константы, такие, что
\begin{equation*}
	0 \not= \alpha_1 \not= \alpha_2 \not= ... \not= \alpha_p \not= 0.
\end{equation*}
Точки $\tau_0$, $\tau_1$, $\tau_2$, ..., $\tau_{p - 1}$, $\tau_p$ задают разбиение отрезка $[0,T]$ по правилу
\begin{equation*}
	0 = \tau_0 < \tau_1 < \tau_2 < ... < \tau_{p - 1} < \tau_p = T.
\end{equation*}

Назовём задачу \eqref{nonlocal problem} нелокальной задачей.

\subsection{Редуцированная задача}

Итак, рассматриваем задачу \eqref{nonlocal problem}. Положим в нелокальном условии \eqref{integral averaging} \linebreak $x = 0$. 
Получим условие согласования
\begin{equation} \label{int_eta_gamma = psi(0)}
	\int\limits_{0}^{T} \eta(t) \gamma(t)\,dt = \psi(0).
\end{equation}
Равенство \eqref{int_eta_gamma = psi(0)} есть необходимое условие разрешимости нелокальной задачи.

Решение задачи \eqref{nonlocal problem} представим в виде
\begin{equation*}
	u(x,t) = v(x,t) + w(x,t),
\end{equation*}
где функция $v(x,t)$ неизвестна, а $w(x,t)$ задаётся в явном виде
\begin{equation} \label{w(x,t)}
w(x,t) = 
	\begin{cases}
	\; \gamma\left(\,t - \dfrac{x}{a}\,\right), & \qquad x < at, \\[3mm]
	\; \gamma(0), & \qquad x \ge at.
	\end{cases}
\end{equation}
Так определённая функция $w(x,t)$ является решением задачи
\begin{equation*}
\begin{cases}
	\; w_t + aw_x + \sigma(x)w = 0, \qquad 0 \le x \le l, \quad 0 \le t \le T, \\[1mm]
	\; w(0,t) = \gamma(t), \\[1mm]
	\; w(x,0) = \gamma(0).
\end{cases}
\end{equation*}

Для $v(x,t)$ получаем задачу
\begin{equation*}
\begin{cases}
	\; v_t + av_x + \sigma(x)v = 0, \qquad 0 \le x \le l, \quad 0 \le t \le T, \\[3mm]
	\; v(0,t) = 0, \\[1mm]
	\; \int\limits_{0}^{T} \eta(t) v(x,t)\,dt = \widetilde{\psi}(x).
\end{cases}
\end{equation*}
Здесь 
\begin{equation*}
	\widetilde{\psi}(x) = \psi(x) - \int\limits_{0}^{T} \eta(t) w(x,t)\,dt, \qquad 0 \le x \le l,
\end{equation*}
есть новая заданная функция.
\goodbreak
Прежде чем изучать поставленную нелокальную задачу для уравнения переноса \eqref{transport equation},
выясним, как обстоят дела с аналогичными нелокальными задачами для дифференциальных уравнений в банаховом пространстве.

\newpage

\section*{Основная часть}
\addcontentsline{toc}{section}{Основная часть}
\section{Абстрактная задача}

\subsection{Основные определения}
Всюду далее считаем, что $E$~--- банахово пространство. Рассматриваем линейный оператор $A$, действующий в пространстве $E$.
Область определения оператора $A$ обозначим через $D(A)$, а множество значений~--- через $R(A)$.
\begin{definition} \label{linear manifold}
	Подмножество $L$ линейного пространства называется линейным многообразием, если
	\begin{equation*}
		x, y \in L \quad \Longrightarrow \quad c_1x + c_2y \in L, \; \forall c_1, c_2 \in \mathbb{R}.
	\end{equation*}
\end{definition}

\begin{definition} \label{closed operator}
	Оператор $A$ такой, что $ D(A) $~--- линейным многообразие,
	называется \textit{замкнутым}, если $\forall \{x_n\}  \in D(A)$ справедлива следующая импликация:
	\begin{equation*}
		x_n \rightarrow x, \; Ax_n \rightarrow y, \quad n \rightarrow \infty  \quad \Longrightarrow \quad x \in D(A), \; Ax = y. 
	\end{equation*}
\end{definition}

В силу этого определения будем априори считать, 
что у замкнутого оператора область определения является линейным многообразием.

\begin{definition}
	Оператор $A$ \textit{непрерывно обратим}, если $R(A) = E$, оператор $A$ обратим и $A^{-1}$ ограничен.
\end{definition}

	Далее полагаем $\lambda$~--- действительное число, $I$~--- единичный оператор в $E$.
	
\begin{definition}
	Точка $\lambda$ называется \textit{регулярной} точкой оператора $A$, если оператор $A - \lambda I$ непрерывно обратим.
\end{definition}

\begin{definition}
	Совокупность регулярных точек оператора $A$ называется \textit{резольвентным множеством} оператора $A$
	и обозначается $\rho(A)$.
\end{definition}

\begin{definition} \label{resolvent}
	Если $\lambda \in \rho(A)$, то ограниченный линейный оператор \linebreak $R(\lambda; A) = (A - \lambda I)^{-1}$ называется 
	\textit{резольвентой} оператора $A$.
\end{definition}

\begin{definition} \label{semigroup}
	Однопараметрическое семейство ограниченных линейных операторов $ U(t): E \rightarrow E $, $\, t \in [0, +\infty) $ 
	называется полугруппой ограниченных линейных операторов в $E$ или просто \textit{полугруппой}, если \\
	1) $ U(0) = I $; \\
	2) $ U(t + s) = U(t)U(s), \qquad \forall t, s \ge 0 $.
\end{definition}

\begin{definition} \label{strongly continious semigroup}
	Полугруппа $ U(t)$ в пространстве $ E $ называется \textit{сильно непрерывной}, если 
	\begin{equation*}
		\lim\limits_{t \rightarrow 0+} U(t)x = x, \qquad \forall x \in E.
	\end{equation*}
	Такую полугруппу также называют полугруппой класса $ C_0 $ или просто \linebreak $ C_0\, $-полугруппой.
\end{definition}

\begin{definition} \label{generator}
	Оператор $A$ с областью определения
	\begin{equation*}
		D(A) = \left\{ x \in E \; \bigg| \quad \exists \lim\limits_{t \rightarrow 0+} \frac{U(t)x - x}{t} \right\} 
	\end{equation*}
	и такой, что
	\begin{equation*}
		Ax = \lim\limits_{t \rightarrow 0+} \frac{U(t)x - x}{t} = \left. \frac{d^+}{dt}\,[\, U(t) x \,] \,\right|_{t = 0}, 
		\qquad \forall x \in D(A),
	\end{equation*}
	называется \textit{производящим оператором} полугруппы $U(t)$ (или просто \textit{генератором} полугруппы $U(t)$).
	Также говорят, что оператор $ A $ \textit{порождает} полугруппу $U(t)$.
\end{definition}

\begin{definition} \label{quasi-nilpotent semigroup}
	Полугруппа $U(t)$ класса~$C_0$ называется \textit{квазинильпотентной} (или \textit{суперустойчивой}), 
	если она имеет бесконечный отрицательный экспоненциальный тип:
	\begin{equation*}
		\omega_0 \equiv \lim\limits_{t \rightarrow +\infty} \frac{\ln \|\, U(t) \, \|}{t} = -\infty.
	\end{equation*}
\end{definition}

\begin{definition} \label{nilpotent semigroup}
	Полугруппа $U(t)$ называется \textit{нильпотентной}, если \linebreak $\exists \, t_0 > 0$ такое, что
	\begin{equation*}
		U(t) = 0, \qquad \forall t \ge t_0.
	\end{equation*}
\end{definition}
\subsection{Свойства производящего оператора полугруппы}
\theoremstyle{definition}
\begin{statement} \label{U(t)x in D(A)}
	Пусть оператор $A$ порождает полугруппу $U(t)$ класса~$C_0$. Тогда,
	если $ x \in D(A) $, то $ U(t)x \in D(A), \; \forall t \ge 0. $
\end{statement}
\begin{proof}
	По определению
	\begin{equation*}
		Ax = \lim\limits_{s \rightarrow 0+} \frac{U(s)x - x}{s}.
	\end{equation*}
	При каждом фиксированном $t \ge 0$ оператор $U(t)$ ограничен и линеен, следовательно, непрерывен. С учётом изложенного получаем
	\begin{gather*}
		AU(t)x = \lim\limits_{s \rightarrow 0+} \frac{U(s)U(t)x - U(t)x}{s} = 
		\lim\limits_{s \rightarrow 0+} U(t)\frac{U(s)x - x}{s} = \\[3mm]
		= U(t)\lim\limits_{s \rightarrow 0+} \frac{U(s)x - x}{s} = U(t)Ax.
	\end{gather*}
	При каждом фиксированном $t \ge 0$ элемент $U(t)Ax$ существует, поскольку $x \in D(A)$, $Ax \in E$.
\end{proof}

\begin{statement} \label{A = d/dx}
	Пусть оператор $A$ порождает полугруппу $U(t)$ класса~$C_0$. Тогда 
	справедливо следующее равенство
	\begin{equation} \label{integral identity}
		A \int\limits_{t_1}^{t_2} U(s)x\,ds = U(t_2)x - U(t_1)x, \qquad \forall t_1, t_2 \ge 0, \quad \forall x \in E.
	\end{equation}
\end{statement}

\begin{proof}
	Для начала докажем
	\begin{equation} \label{middle}
		\lim\limits_{h \rightarrow 0+} \frac{1}{h} \int\limits_{t}^{t + h} U(s)x\,ds = U(t)x, 
		\qquad \forall t \ge 0, \quad  \forall x \in E.
	\end{equation}
	Фиксируем $h > 0$, $x \in E$. Тогда
	\begin{gather*}
		\frac{1}{h} \int\limits_{t}^{t + h} U(s)x\,ds - U(t)x = 
		\frac{1}{h} \int\limits_{t}^{t + h} \h[\, U(s) - U(t) \,\h]x\,ds = \\[1mm]
		= U(t) \int\limits_{t}^{t + h} \frac{U(s - t) - I}{h}\,x\,ds =
		U(t) \int\limits_{0}^{h} \frac{U(s) - I}{h}\,x\,ds = \\[1mm]
		= U(t) \int\limits_{0}^{h} \frac{s}{h} \; \frac{U(s) - I}{s}\,x\,ds.
	\end{gather*}
	Оценим последнее выражение.
	\begin{equation*}
		\left\|\, U(t) \int\limits_{0}^{h} \frac{s}{h} \; \frac{U(s) - I}{s}\,x\,ds \,\right\|_E
		\le \,\|\, U(t) \,\|_{\mathfrak{L}(E)} \int\limits_{0}^{h} \, \left\|\, \frac{U(s) - I}{s}\,x \,\right\|_E ds.
	\end{equation*}
	При $ h \rightarrow 0+ $ получим 
	\begin{equation*}
		\|\, U(t) \,\|_{\mathfrak{L}(E)} \int\limits_{0}^{0} \, \left\|\, Ax \,\right\|_E ds = 0,
	\end{equation*}
	что и доказывает равенство \eqref{middle}. 
	
	Вернёмся к доказательству утверждения. Фиксируем $h > 0$, $t_1, t_2 \ge 0$, $x \in E$. 
	Рассмотрим следующее выражение:
	\begin{equation} \label{in middle}
		\frac{U(h) - I}{h} \int\limits_{t_1}^{t_2} U(s)x \,ds = \frac{1}{h} 
		\left(\, \int\limits_{t_1}^{t_2} U(s + h)x \,ds - \int\limits_{t_1}^{t_2} U(s)x \,ds \,\right).
	\end{equation}
	Преобразуем первый интеграл в правой части равенства \eqref{in middle}.
	\begin{equation*}
		\int\limits_{t_1}^{t_2} U(s + h)x \,ds = \int\limits_{t_1 + h}^{t_2 + h} U(s)x \,ds\, = 
		\int\limits_{t_2}^{t_2 + h} U(s)x \,ds + \int\limits_{t_1}^{t_2} U(s)x \,ds + \int\limits_{t_1 + h}^{t_1} U(s)x \,ds.
	\end{equation*}
	Тогда в соотношении \eqref{in middle} имеем
	\begin{equation*}
		\frac{U(h) - I}{h} \int\limits_{t_1}^{t_2} U(s)x \,ds =  
		\frac{1}{h} \int\limits_{t_2}^{t_2 + h} U(s)x \,ds - \frac{1}{h} \int\limits_{t_1}^{t_1 + h} U(s)x \,ds.
	\end{equation*}
	Последнее выражение в силу \eqref{middle} стремится к $ U(t_2)x - U(t_1)x $ при $ h \rightarrow 0+ $. С другой стороны,
	имеем
	\begin{equation*}
		\lim\limits_{h \rightarrow 0+} \frac{U(h) - I}{h} \int\limits_{t_1}^{t_2} U(s)x \,ds = A\int\limits_{t_1}^{t_2} U(s)x \,ds,
	\end{equation*}
	откуда и следует равентсво \eqref{integral identity}.
\end{proof}

\subsection{Общая постановка нелокальной задачи}
В вещественном банаховом пространстве $E$ рассматривается дифференциальное уравнение
\begin{equation} \label{abstract equation}
	\frac{du(t)}{dt} = Au(t), \qquad t \ge 0,
\end{equation}
с замкнутым линейным оператором $A$. Область определения $D(A)$ плотна в~$E$. Оператор $A$ порождает в $E$ нильпотентную полугруппу $U(t)$ класса~$C_0$.

Поскольку нильпотентная полугруппа $U(t)$, очевидно, квазинильпотентна, справедливо равенство
[см. определение \ref{quasi-nilpotent semigroup}]
\begin{equation*}
	\lim\limits_{t \rightarrow +\infty} \frac{\ln \|\, U(t) \, \|}{t} = -\infty.
\end{equation*}
Тогда, согласно \cite[теорема VIII.1.11]{Dunford_Schwartz}, резольвента $R(\lambda; A)$ определена при 
$\forall \lambda \in \mathbb{R}$ по формуле
\begin{equation*}
	R(\lambda; A)x = \int\limits_{0}^{+\infty}e^{-\lambda t}\,U(t)x\,dt, \qquad \forall x \in E.
\end{equation*}
Положим в этой формуле $\lambda = 0$. Получим
\begin{equation*}
	A^{-1}x = \int\limits_{0}^{+\infty}U(t)x\,dt, \qquad \forall x \in E.
\end{equation*}
Таким образом, оператор $A^{-1}$ существует и определён определён на всём $E$.

Обобщённым решением уравнения \eqref{abstract equation} назовём векторную функцию \linebreak 
$ u(t) = U(t)u_0 $, заданную при $ t \ge 0 $, с элементом $ u_0 \in E $. 
При этом $ u_0 = u(0) $ есть начальное состояние решения. 
В случае, когда $ u_0 \in D(A) $, решение $ u(t) = U(t)u_0 $ называем классическим.

Отметим, что так определённое обощённое решение $u(t) = U(t)u_0$ есть векторная функция из класса 
$C([0, +\infty); E)$, удовлетворяющая проинтегрированной версии уравнения \eqref{abstract equation} в том смысле, что
\begin{equation*}
	u(t_2) - u(t_1) = A\int\limits_{t_1}^{t_2} u(t) \,dt, \qquad \forall t_1, t_2 \ge 0.
\end{equation*}
Последнее равенство в наших предположениях заведомо выполнено в силу утверждения \ref{A = d/dx}.

Что касается классичекого решения, оно удовлетворяет уравнению \eqref{abstract equation} в строгом смысле,
являясь функцией из класса $C^1([0, +\infty); E)$ со значениями в $D(A)$.

В качестве дополнительного условия возьмём интеграл
\begin{equation} \label{abstarct integral}
	\int\limits_{0}^{T} \eta(t)u(t)\,dt = \psi.
\end{equation}
Здесь элемент $ \psi \in E $ задан, функция $ \eta(t) $ известна, кусочно постоянна и определяется по формуле \eqref{eta(t)}.

Обобщённым решением задачи \eqref{abstract equation}, \eqref{abstarct integral} назовём векторную функцию \linebreak $ u(t) = U(t)u_0 $, 
где элемент $ u_0 \in E $ выбран так, что выполнено условие \eqref{abstarct integral}. Если $ u_0 \in D(A) $, решение 
$ u(t) = U(t)u_0 $ называем классическим.

Поставленную задачу \eqref{abstract equation}--\eqref{abstarct integral} коротко называем абстрактной задачей.

\subsection{Вывод операторного уравнения} %\label{inference of operator equation}
Подставим $ u(t) = U(t)u_0 $ в \eqref{abstarct integral}. Получим 
\begin{equation*}
	\int\limits_{0}^{T} \eta(t)U(t)u_0\,dt = \psi.
\end{equation*}
Учитывая конкретный вид функции $\eta(t)$ [см. формула \eqref{eta(t)}], получим
\begin{equation*}
	\sum\limits_{k = 1}^{p} \alpha_k \int\limits_{\tau_{k - 1}}^{\tau_k} U(t)u_0\,dt = \psi.
\end{equation*}
Предположим $\psi \in D(A)$. Подействуем на обе части равенства оператором $(-A)$. В силу утверждения \ref{A = d/dx} имеем
\begin{equation*} 
	-\sum\limits_{k = 1}^{p} \alpha_k \,\h[\, U(\tau_k) - U(\tau_{k - 1}) \,\h]\, u_0 = -A\psi.
\end{equation*}
Далее, применим преобразование Абеля, обозначив $ \alpha_{p + 1} = 0 $.
\begin{equation*}
\begin{aligned}
	  & \sum\limits_{k = 1}^{p} \alpha_k \,\h[\, U(\tau_k) - U(\tau_{k - 1}) \,\h] = \sum\limits_{k = 1}^{p} \alpha_k U(\tau_k) - 
		\sum\limits_{k = 1}^{p} \alpha_k U(\tau_{k - 1}) = \\[1mm]
	= & \sum\limits_{k = 1}^{p} \alpha_k U(\tau_k)\, - \sum\limits_{k = 0}^{p - 1} \alpha_{k + 1} U(\tau_k) = 
	\sum\limits_{k = 1}^{p} \alpha_k U(\tau_k)\, - \\[1mm]
	- & \sum\limits_{k = 1}^{p} \alpha_{k + 1} U(\tau_k) - \alpha_1 U(\tau_0) = 
		-\alpha_1 + \sum\limits_{k = 1}^{p} (\alpha_k - \alpha_{k + 1})U(\tau_k).
\end{aligned}
\end{equation*}
Таким образом
\begin{equation*}
	\alpha_1 u_0 - \sum\limits_{k = 1}^{p} (\alpha_k - \alpha_{k + 1})U(\tau_k)u_0 = -A\psi.
\end{equation*}
Обозначим $ B = \sum\limits_{k = 1}^{p} (\alpha_k - \alpha_{k + 1})U(\tau_k), \; g = -A\psi $. Получим операторное уравнение
\begin{equation} \label{operator equation}
	\alpha_1 u_0 - Bu_0 = g.
\end{equation}

\subsection{Разрешающая формула}
Заметим, что оператор $B$ является линейной комбинацией значений нильпотентной полугруппы $U(\tau_k)$ при $k = 1, ..., p$.
В свою очередь, оператор $B^n$ будет являться линейной комбинацией выражений вида
\begin{equation*}
	U(\tau_{k_1})U(\tau_{k_2})...U(\tau_{k_n\!}) = U\!\left( \sum_{i = 1}^{n}\tau_{k_i} \right), 
	\quad k_i \in \{1,...,p\}, \; i = 1,...,n.
\end{equation*}
Поскольку $\tau_{k_i} > 0$, найдётся такое $n \in \mathbb{N}$, что 
\begin{equation*}
	\sum_{i = 1}^{n}\tau_{k_i} \ge t_0, \quad \forall (k_1,...,k_n) : k_i \in \{1,...,p\}, \; i = 1,...,n.
\end{equation*}
В силу определения $U(t)$ это означает, что $B^n = 0$, то есть оператор $B$ является нильпотентным. 

Прежде чем найти индекс нильпотентности оператора $B$, выведем формулу для оператора $B^n$, $n \ge 2$. 
Обозначим
\begin{equation*}
	C_n^{\,k_1, ..., k_p} \equiv \frac{n!}{k_1! \cdot\cdot\cdot k_p!}.
\end{equation*}
Учитывая, что $\alpha_{p + 1} = 0$, имеем
\begin{equation*}
\begin{aligned}
	  & B^n = \left[\, \sum\limits_{k = 1}^{p} (\alpha_k - \alpha_{k + 1})U(\tau_k) \,\right]^n = \\[2mm]
	= & \sum\limits_{\substack{{k_i \ge 0} \\_{k_1 + ... + k_p = n}}} C_n^{\,k_1, ..., k_p}\,
	 \h[\, (\alpha_1 - \alpha_2)U(\tau_1) \, \h]^{k_1}\; \times \\[2mm] 
\times & \;\h[\, (\alpha_2 - \alpha_3)U(\tau_2) \, \h]^{k_2}\, ... \;\h[\, (\alpha_p - \alpha_{p + 1})U(\tau_p) \, \h]^{k_p} = \\[2mm]
	= & \sum\limits_{\substack{{k_i \ge 0} \\_{k_1 + ... + k_p = n}}} C_n^{\,k_1, ..., k_p}\,
	(\alpha_1 - \alpha_2)^{k_1}\,U(k_1\tau_1)\; \times \\[2mm]
\times & \;(\alpha_2 - \alpha_3)^{k_2}\,U(k_2\tau_2)\, ... \,(\alpha_p - \alpha_{p + 1})^{k_p}\,U(k_p\tau_p) = \\[2mm]
	= & \sum\limits_{\substack{{k_i \ge 0} \\_{k_1 + ... + k_p = n}}} C_n^{\,k_1, ..., k_p}\, (\alpha_1 - \alpha_2)^{k_1}\,
	(\alpha_2 - \alpha_3)^{k_2}\, ... \,(\alpha_p - \alpha_{p + 1})^{k_p}\,U\!\left( \sum\limits_{i = 1}^{p}k_i\tau_i \right).
\end{aligned}
\end{equation*}
Таким образом 
\begin{equation} \label{B^n}
	B^n = \sum\limits_{\substack{{k_i \ge 0} \\_{k_1 + ... + k_p = n}}} C_n^{\,k_1, ..., k_p}\,
	\prod_{m = 1}^{p}(\alpha_m - \alpha_{m + 1})^{k_m}\,U\!\left( \sum\limits_{i = 1}^{p}k_i\tau_i \right).
\end{equation}

Обозначим
\begin{equation*}
S^p_n = \left\{\,(k_1, ..., k_p) \, \left| \; k_i \in \mathbb{N}_0, \; i = 1,...,p; \; \sum_{i = 1}^{p}k_i = n \right. \,\right\}.
\end{equation*}

Найдём $n_0$~--- индекс нильпотентности оператора $B$.
Из фомрулы \eqref{B^n} вытекает, что число $n_0$ определяется следующим образом:
\begin{equation*}
	\sum\limits_{i = 1}^{p} k_i \tau_i \ge t_0, \quad \forall (k_1,...,k_p) \in S^p_{n_0}.
\end{equation*}
%$ \sum\limits_{i = 1}^{p} k_i \tau_i \ge t_0, $ для всех таких наборов $ k_1, ..., k_p $, что 
%$ \sum\limits_{i = 1}^{p} k_i = n_0, \; k_j \ge 0, \; j = 1, ..., p \, $, 
Здесь $ t_0 > 0 $ из определения \ref{nilpotent semigroup}. Это условие выполнено тогда и только тогда, когда
\begin{equation} \label{functional}
	\min_{(k_1, ..., k_p) \in S^p_{n_0}} \sum\limits_{i = 1}^{p} k_i \tau_i \ge t_0.
\end{equation}
Поскольку $\tau_{i - 1} < \tau_i, \; i = 1,...,p$, минимум в \eqref{functional} достигается 
при $ k_1 = n_0, \; k_2 = ... = k_p = 0 $. Следовательно,
\begin{onehalfspacing}
	\begin{equation*}
	n_0 \tau_1 \ge t_0 \quad \Longrightarrow \quad n_0 \ge \, \frac{t_0}{\tau_1} \quad \Longrightarrow \quad 
	n_0 = \left\lceil \frac{t_0}{\tau_1} \right\rceil.
	\end{equation*}
\end{onehalfspacing}
Итак, оператор $B$ имеет индекс нильпотентности $ n_0 = \left\lceil t_0 / \tau_1 \right\rceil $. 

Найдём начальное состояние $u_0$. Для этого рассмотрим операторное уравнение \eqref{operator equation}.
Разделим обе части равенства на $\alpha_1$. 
Обозначим $\widetilde{B} = B/\alpha_1$, \linebreak  $\widetilde{g} = g/\alpha_1$. Имеем
\begin{equation} \label{u_0}
	u_0 - \widetilde{B}u_0 = \widetilde{g}.
\end{equation}
Последовательно подействуем операторами $ \widetilde{B}^{\,n}, \; n = 1, 2, ..., n_0 - 1 ,$ на равенство \eqref{u_0}. Получим
\begin{gather*} 
	\widetilde{B}u_0 - \widetilde{B}^2u_0 = \widetilde{B}\,\widetilde{g}, \\
	\widetilde{B}^2u_0 - \widetilde{B}^3u_0 = \widetilde{B}^2\,\widetilde{g}, \\
	.\phantom{-}.\phantom{-}.\phantom{-}.\phantom{-}. \\
	\widetilde{B}^{n_0 - 2}u_0 - \widetilde{B}^{n_0 - 1}u_0 = \widetilde{B}^{n_0 - 2}\,\widetilde{g}, \\
	\widetilde{B}^{n_0 - 1}u_0  = \widetilde{B}^{n_0 - 1}\,\widetilde{g}.
\end{gather*}
Складывая полученные равенства с равенством \eqref{u_0}, получим
\begin{equation} \label{solution}
	u_0 = \sum\limits_{n = 0}^{n_0 - 1} \widetilde{B}^{n}\,\widetilde{g} = 
	\sum\limits_{n = 0}^{n_0 - 1} \frac{1}{\alpha_1^{\,\,n + 1}}B^{n} g.
\end{equation}
С учётом формулы \eqref{B^n} и обозначения элемента $g$, окончательно получим
\begin{onehalfspacing}
	\begin{equation} \label{abstract solution}
		u_0 = \sum\limits_{n = 0}^{n_0 - 1} \frac{1}{\alpha_1^{\,\,n + 1}}\sum\limits_{\substack{{k_i \ge 0} \\_{k_1 + ... + k_p = n}}}\!\!\!\!C_n^{\,k_1, ..., k_p}\,
		\prod_{m = 1}^{p}(\alpha_m - \alpha_{m + 1})^{k_m}\,U\!\left( \sum\limits_{i = 1}^{p}k_i\tau_i \right)(-A\psi),
	\end{equation}
\end{onehalfspacing}
где $n_0 = \left\lceil \dfrac{t_0}{\tau_1} \right\rceil$, $ \alpha_{p + 1} = 0 $.

\subsection{Важный частный случай}
Рассмотрим формулу \eqref{abstract solution} при $p = 2$. 
Обозначим $ \tau_1 = \tau $. С учётом $\tau_2 = T$, имеем 
\begin{gather*}
	u_0 = \sum\limits_{n = 0}^{n_0 - 1} \frac{1}{\alpha_1^{\,\,n + 1}}\sum\limits_{\substack{{k_i \ge 0} \\_{k_1 + k_2 = n
	}}}C_n^{\,k_1, ..., k_2}\,
	\prod_{m = 1}^{2}(\alpha_m - \alpha_{m + 1})^{k_m}\,U\!\left( \sum\limits_{i = 1}^{2}k_i\tau_i \right)(-A\psi) = \\[2mm]
	= \sum\limits_{n = 0}^{n_0 - 1} \frac{1}{\alpha_1^{\,\,n + 1}}
	\sum\limits_{k = 0}^n C_n^{k} (\alpha_1 - \alpha_2)^k\, \alpha_2^{n - k}\,U(\,k\tau + [n - k]T\,) (-A\psi) = \\[2mm]
	= \frac{1}{\alpha_1}\sum\limits_{n = 0}^{n_0 - 1} \left(\frac{\alpha_2}{\alpha_1}\right)^{\!\!n} \;
	\sum\limits_{k = 0}^n C_n^{k} \left(\frac{\alpha_1}{\alpha_2} - 1\right)^{\!k} \! U(\,k\tau + [n - k]T\,) (-A\psi).
\end{gather*}
Обозначим $ \alpha \equiv \alpha_1 $, $ r \equiv \alpha_1/\alpha_2 $. В итоге, получим
\begin{equation} \label{particular case}
	u_0 = \frac{1}{\alpha}\sum\limits_{n = 0}^{n_0 - 1} 
	\left( \frac{1}{r} \right)^{\!\!n}\,
	\sum_{k = 0}^{n}C_n^k \left(\, r - 1 \,\right)^k U(\, k\tau + [n - k]T \,)\, (-A\psi).
\end{equation}
Формула \eqref{particular case} есть частный случай общего соотношения \eqref{abstract solution}. Этот вариант
удобно применять в случае, когда весовая функция $\eta(t)$ имеет лишь одну точку разрыва.

\subsection{Характер получаемых решений}
Прежде всего отметим, что абстрактная задача неразрешима при \linebreak $ \psi \in E \, \textbackslash \, D(A) $. 
Действительно, подставим $ u(t) = U(t)u_0 $ в \eqref{abstarct integral} и воспользуемся определением $\eta(t)$. Получим
\begin{equation} \label{sum_int = psi}
	\sum\limits_{k = 1}^{p} \alpha_k \int\limits_{\tau_{k - 1}}^{\tau_k} U(t)u_0\,dt = \psi.
\end{equation}
Так как оператор A порождает полугруппу $U(t)$ класса~$C_0$, справедливо утвержедене \ref{A = d/dx}. 
Следовательно,
\begin{equation*}
	\int\limits_{\tau_{k - 1}}^{\tau_k} U(t)u_0\,dt \in D(A), \qquad k = 1, ..., p.
\end{equation*}
В силу линейности оператора A
\begin{equation*}
	\sum\limits_{k = 1}^{p} \alpha_k \int\limits_{\tau_{k - 1}}^{\tau_k} U(t)u_0\,dt \in D(A).
\end{equation*}
Отсюда следует неразрешимость абстрактной задачи при $\psi \in E \setminus D(A)$.

Теперь предположим, что $\psi \in D(A)$. Тогда решение абстрактной задачи задаётся формулой \eqref{abstract solution}, 
что и означает разрешимость абстрактной задачи.

Пусть абстрактная задача разрешима. Тогда она эквивалентна операторному уравнению \eqref{operator equation} в том смысле, 
что множества их решений совпадают. Действительно, учитывая обратмость оператора $A$,
все преобразования при получении операторного уравнения \eqref{operator equation} являются
эквивалентными. 

Выясним, при каких $ \psi \in D(A) $ задача будет иметь классическое решение, то есть будет выполнено условие $ u_0 \in D(A) $.
Рассмотрим операторное уравнение \eqref{operator equation} с учётом обозначений.
\begin{equation*}
	\alpha_1 u_0 - \sum\limits_{k = 1}^{p} (\alpha_k - \alpha_{k + 1})U(\tau_k)u_0 = -A\psi.
\end{equation*}
Положим в этом уравнении $u_0 \in D(A)$. 
Тогда, в силу утверждения \ref{U(t)x in D(A)}, имеем \linebreak $U(\tau_k)u_0 \in D(A)$, $k = 1,..., p$. 
Следовательно, 
\begin{equation*}
	\alpha_1 u_0 - \sum\limits_{k = 1}^{p} (\alpha_k - \alpha_{k + 1})U(\tau_k)u_0 \in D(A).
\end{equation*}
Тогда
\begin{equation*}
	A\psi \in D(A) \quad \Longleftrightarrow \quad \psi \in A^{-1}(D(A)) = D(A^2),
\end{equation*}
так как оператор $A^2$ действует по схеме $ A^{-1}(D(A)) \xrightarrow{A} D(A) \xrightarrow{A} R(A). $

В итоге, \\
1) при $ \psi \in E \, \textbackslash \, D(A) $ абстрактная задача неразрешима; \\
2) при $ \psi \in D(A) $ решение абстрактной задачи является обобщённым; \\
3) при $ \psi \in D(A^2) $ решение абстрактной задачи является классическим.

\newpage

\section{Пример: оператор простого переноса}
Перейдём к конкретным примерам. Пусть $E = L_1[0,l]$ с обычной лебеговой нормой
\begin{equation} \label{norm}
	\|\, f \,\| = \int\limits_0^l | f(x) | \,dx.
\end{equation}
Число $l > 0$ считаем фиксированным.

В пространстве $E = L_1[0,l]$ рассмотрим оператор простого переноса
\begin{equation} \label{simple transport operator}
	A = -a\dfrac{d}{dx}
\end{equation}
с фиксированным числом $a > 0$. Считаем, что оператор \eqref{simple transport operator} имеет область определения
\begin{equation} \label{domain of simple transport operator}
	D(A) = \{\, f \in AC[0,l] \;\, | \;\, f(0) = 0 \,\},
\end{equation} 
где $AC[0,l]$~--- пространство абсолютно непрерывных на отрезке $[0,l]$ функций [см. \cite{Kolmogorov_Fomin}, стр. 342-347]. 

Полугруппа $U(t)$, порождаемая оператором $A$, имеет вид 
\begin{equation*}
U(t)f(x) = 
	\begin{cases}
		\,f(x - at), & \qquad x > at, \\
		\,0, 		 & \qquad x \le at,
	\end{cases}
\end{equation*}
или
\begin{equation*}
	U(t)f(x) = \Theta(x - at)f(x - at), \qquad 0 \le x \le l, \quad 0 \le t \le T.
\end{equation*}
Здесь и далее $\Theta(x)$~--- функция Хевисайда, определённая следующим образом:
\begin{equation*}
\Theta(x) = 
	\begin{cases}
		\, 1, & \quad x > 0, \\
		\, 0, & \quad x \le 0.
	\end{cases}
\end{equation*} 
Отметим, что число $t_0$ из определения нильпотентной полугруппы \eqref{nilpotent semigroup} в данном случае равно $x/a$, поскольку действие полугруппы рассматривается при фиксированном $x$.

В этом случае разрешающая формула \eqref{abstract solution} имеет вид
\begin{equation} \label{homogenous solution}
\begin{gathered} 
	u_0(x) = a\,\sum\limits_{n = 0}^{n_0 - 1} \frac{1}{\alpha_1^{\,\,n + 1}}
	\sum\limits_{\substack{{k_i \ge 0} \\_{k_1 + ... + k_p = n}}}\!\!\!\!C_n^{\,k_1, ..., k_p}\;
	\prod_{m = 1}^{p}(\alpha_m - \alpha_{m + 1})^{k_m} \,\times \\[2mm] \times\,
	\Theta\!\left(x - a\sum\limits_{i = 1}^{p}k_i\tau_i \right)\psi'\!\left(x - a\sum\limits_{i = 1}^{p}k_i\tau_i\right), 
	\qquad 0 \le x \le l,
\end{gathered}
\end{equation}
где $n_0 = \lceil x / (a\tau_1) \rceil$, $\;\alpha_{p + 1} = 0$.

\newpage

\section{Пример: оператор переноса с поглощением}
Пусть значение $l > 0$ фиксировано. В пространстве \mbox{$E = L_1[0,l]$} с лебеговой нормой рассмотрим \eqref{norm} 
оператор переноса с поглощением
\begin{equation} \label{absorbing transport operator}
	A = -a\,\dfrac{d}{dx} - \sigma.
\end{equation}
Здесь коэффициенты 
\begin{equation*}
	a \in \mathbb{R}_+ \setminus \{0\}, \quad\sigma \in \{\, f \in C[0,l] \;\, | \;\, f(x) \ge 0, \; x \in [0,l] \,\}
\end{equation*}
фиксированы. Оператор \eqref{absorbing transport operator} имеет область определения 
\begin{equation*}
	D(A) = \{\, f \in AC[0,l] \;\, | \;\, f(0) = 0 \,\}.
\end{equation*}

Полугруппа $U(t)$, порождённая оператором $A$, имеет вид 
\begin{equation*}
U(t)f(x) = 
	\begin{cases}
		\,f(x - at)\exp\left(\, -\frac{1}{a}\int\limits_{0}^{at}\sigma(x - s) \,ds \,\right), & \qquad x > at, \\
		\,0, & \qquad x \le at.
	\end{cases}
\end{equation*}
или
\begin{equation}
\begin{aligned}
	& U(t)f(x) = \Theta(x - at)f(x - at)\exp\left(\, -\frac{1}{a}\int\limits_{0}^{at}\sigma(x - s) \,ds \,\right), \\[3mm]
	& 0 \le x \le l, \quad 0 \le t \le T.
\end{aligned}	
\end{equation}

В этом случае разрешающая формула \eqref{abstract solution} имеет вид
\begin{equation} \label{non-homogenous solution}
\begin{aligned}
	& u_0(x) = a\,\sum\limits_{n = 0}^{n_0 - 1} \frac{1}{\alpha_1^{\,\,n + 1}}
	\sum\limits_{\substack{{k_i \ge 0} \\_{k_1 + ... + k_p = n}}}\!\!\!\!C_n^{\,k_1, ..., k_p}\; 
	\prod_{m = 1}^{p}(\alpha_m - \alpha_{m + 1})^{k_m} \,\times \\[2mm] \times\,
	& \Theta\!\left(x - a\sum\limits_{i = 1}^{p}k_i\tau_i \right)\psi'\!\left(x - a\sum\limits_{i = 1}^{p}k_i\tau_i\right)
	\exp\left(\, -\frac{1}{a}\int\limits_{0}^{a\sum\limits_{i = 1}^{p}k_i\tau_i}\sigma(x - s) \,ds \,\right), \\[3mm] 
	& 0 \le x \le l,
\end{aligned}
\end{equation}
где $n_0 = \lceil x / (a\tau_1) \rceil$, $\;\alpha_{p + 1} = 0$.

\newpage

\section{Описание программы}

\subsection{Краткий обзор}
Рассматриваем нелокальную задачу для уравнения простого переноса.
\begin{equation} \label{nonlocal problem with sigma = 0}
\begin{cases}
	\; u_t + au_x = 0, \qquad 0 \le x \le l, \quad 0 \le t \le T, \\[3mm]
	\; u(0,t) = \gamma(t), \\[1mm]
	\; \int\limits_{0}^{T} \eta(t) u(x,t)\,dt = \psi(x),
\end{cases}
\end{equation}
$u = u(x,t) = \;?$

В задаче \eqref{nonlocal problem with sigma = 0} нелокальное условие $\psi(x)$ должно удовлетворять двум условиям.
\begin{enumerate}
	\item $\psi(0) = 0$~--- функция из области определения оператора простого \mbox{переноса}.
	\item 
	\begin{equation*}
		\int\limits_{0}^{T} \eta(t) \gamma(t)\,dt = \psi(0) 
	\end{equation*}
	--- условие согласования \eqref{int_eta_gamma = psi(0)}.
\end{enumerate}
Поскольку проверять эти условия, особенно второе, бывает затруднительно, определяется смещённая функция 
\begin{equation} \label{tilde_psi}
	\widetilde{\psi}(x) = \psi(x) - \int\limits_0^T \eta(t)w(x,t) \,dt - \psi(0) + \int\limits_0^T \eta(t)\gamma(t) \,dt,
	\qquad 0 \le x \le l.
\end{equation}
В итоге, программа строит график функции $\varphi(x) = u(x,0)$, где $u(x,t)$~--- решение задачи
\begin{equation} \label{nonlocal problem with tilde_psi}
\begin{cases}
	\; u_t + au_x = 0, \qquad 0 \le x \le l, \quad 0 \le t \le T, \\[3mm]
	\; u(0,t) = \gamma(t), \\[1mm]
	\; \int\limits_{0}^{T} \eta(t) u(x,t)\,dt = \widetilde{\psi}(x),
\end{cases}
\end{equation}
$u = u(x,t) = \;?$

Рассмотрим подробнее функцию $\widetilde{\psi}(x)$. Для её вычисления необходимо посчитать 
интегралы
\begin{equation} \label{integrals}
	\int\limits_0^T \eta(t)\gamma(t) \,dt, \quad \int\limits_0^T \eta(t)w(x,t) \,dt.
\end{equation}
Посчитаем отдельно каждый из интегралов \eqref{integrals}.

Для первого интеграла \eqref{integrals} имеем
\begin{equation} \label{int_eta_gamma}
	\int\limits_0^T \eta(t)\gamma(t) \,dt = \sum\limits_{k = 1}^p \int\limits_{\;\tau_{k - 1}}^{\tau_k} \gamma(t) \,dt =
	\sum\limits_{k = 1}^p G(\tau_{k - 1}, \tau_k).
\end{equation}
Функция
\begin{equation} \label{int_gamma}
	G(t_1, t_2) = \int\limits_{t_1}^{t_2} \gamma(t) \,dt
\end{equation}
определяется путём аналитического вычисления интеграла, называемого в Python "символьным интегрированием".

Прежде чем посчитать второй интеграл \eqref{integrals}, выпишем определение функции $w(x,t)$.
\begin{equation*}
w(x,t) = 
	\begin{cases}
		\; \gamma\left(\,t - \dfrac{x}{a}\,\right), & \qquad x \le at, \\[3mm]
		\; \gamma(0), & \qquad x \ge at.
	\end{cases}
\end{equation*}
Ясно, что нужно рассмотреть интеграл в зависимости от соотношения $x - at$.

Пусть $x < aT$. Точнее, $a\tau_{m - 1} \le x \le a\tau_m$, или $\tau_{m - 1} \le x / a \le \tau_m$. Тогда, с учётом того, что
\begin{equation*}
	\int\limits_{t_1}^{t_2} \gamma\left(\,t - \dfrac{x}{a}\,\right) \,dt = \int\limits_{t_1 - x/a}^{t_2 - x/a} \gamma(t) \,dt = 
	G\left(t_1 - \dfrac{x}{a}, t_2 - \dfrac{x}{a}\right),
\end{equation*}
получим
\begin{equation*} \label{int_eta_w_x<aT}
\begin{aligned}
	& \phantom{--------} \int\limits_0^T \eta(t)w(x,t) \,dt = 
	\sum\limits_{k = 1}^p \int\limits_{\;\tau_{k - 1}}^{\tau_k} w(x,t) \,dt = \\[2mm]
	= & \sum\limits_{k = 1}^{m - 1} \alpha_k \!\!\int\limits_{\tau_{k - 1}}^{\tau_k}\!\! \gamma(0) dt +
	\alpha_m \!\!\left[\, \int\limits_{\tau_{m - 1}}^{x/a}\!\!\! \gamma(0) dt +\!\!
	\int\limits_{x/a}^{\tau_m}\!\! \gamma\!\left(\,t - \dfrac{x}{a}\,\right) dt \right] \!\!+\!
	\sum\limits_{k = m + 1}^{p}\! \alpha_k \!\!\int\limits_{\tau_{k - 1}}^{\tau_k}\!\! \gamma \!\left(\,t - \dfrac{x}{a}\,\right)\! dt = \\[2mm]
	=\; & \gamma(0)\sum\limits_{k = 1}^{m - 1} \alpha_k(\tau_k - \tau_{k - 1}) + 
	\alpha_m \left[\, \gamma(0)\left(\, \dfrac{x}{a} - \tau_{m - 1} \,\right) + 
	G\left(0, \tau_{m} - \dfrac{x}{a}\right) \,\right] + \\[2mm]
	& \phantom{--------} + \sum\limits_{k = m + 1}^{p} \alpha_k \;G\left(\tau_{k - 1} - \dfrac{x}{a}, \tau_{k} - \dfrac{x}{a}\right).
\end{aligned}
\end{equation*}

Теперь пусть $x \ge aT$. Имеем
\begin{equation*} \label{int_eta_w_x>aT}
\begin{aligned}
	& \int\limits_0^T \eta(t)w(x,t) \,dt = \sum\limits_{k = 1}^p \int\limits_{\;\tau_{k - 1}}^{\tau_k} w(x,t) \,dt = \\[2mm]
	= & \sum\limits_{k = 1}^{p} \alpha_k \int\limits_{\tau_{k - 1}}^{\tau_k} \gamma(0) \,dt = 
	\gamma(0)\sum\limits_{k = 1}^{p} \alpha_k(\tau_k - \tau_{k - 1}).
\end{aligned}	
\end{equation*}

В итоге, имеем
\begin{equation} \label{int_eta_w_def}
\int\limits_0^T \!\eta(t)w(x,t) dt =
	\begin{cases}
		\gamma(0)\sum\limits_{k = 1}^{m - 1} \alpha_k(\tau_k - \tau_{k - 1}) + 
		\sum\limits_{k = m + 1}^{p} \alpha_k \;G\left(\tau_{k - 1} - \dfrac{x}{a}, \tau_{k} - \dfrac{x}{a}\right) + \\[7mm]
		+ \alpha_m \!\!\left[\, \gamma(0)\!\left(\, \dfrac{x}{a} - \tau_{m - 1} \,\right) + 
		G\!\left(0, \tau_{m} - \dfrac{x}{a}\right) \,\right]\!, \quad \dfrac{x}{a} \in [\tau_{m - 1}, \tau_m], \\[5mm]
		\gamma(0)\sum\limits_{k = 1}^{p} \alpha_k(\tau_k - \tau_{k - 1}), \quad \dfrac{x}{a} > T.
	\end{cases}
\end{equation} \\

Также в программе используются значения производной функции $\widetilde{\psi}(x)$
\begin{equation} \label{d_tilde_psi}
	\widetilde{\psi}\,'(x) = \psi'(x) - \dfrac{d}{dx}\int\limits_0^T \eta(t)w(x,t) \,dt.
\end{equation}
Ясно, что необходимо посчитать функцию 
\begin{equation*}
	\dfrac{d}{dx}\int\limits_0^T \eta(t)w(x,t) \,dt.
\end{equation*}
Учитывая вид функции $w(x,t)$, будем рассматривать этот интеграл в зависимости от 
соотношения $x - at$.

Пусть $x < aT$. Конкретно, $a\tau_{m - 1} \le x \le a\tau_m$. Согласно \eqref{int_eta_w_def} получим
\begin{equation*}
\begin{aligned}
	& \phantom{----------} \dfrac{d}{dx}\int\limits_0^T \eta(t)w(x,t) \,dt = \\[3mm]
	=\; & \dfrac{d}{dx}\left[\, \alpha_m \!\!\left(\, \gamma(0)\!\left(\, \dfrac{x}{a} - \tau_{m - 1} \,\right) +
	\int\limits_{0}^{\tau_m - x/a}\! \gamma(t) \,dt \right) +
	\sum\limits_{k = m + 1}^{p}\! \alpha_k \!\!\int\limits_{\tau_{k - 1} - x/a}^{\tau_k - x/a}\!\! \gamma(t) \,dt \,\right] = \\[3mm] 
	=\; & \alpha_m \left[\, \gamma(0)\dfrac{1}{a} + \left(-\,\dfrac{1}{a}\right)\gamma\!\left(\tau_m - \dfrac{x}{a}\right) \,\right]
	+ \\[3mm] +\, & \sum\limits_{k = m + 1}^{p} \alpha_k \left(-\,\dfrac{1}{a}\right) \left[\,
	\gamma\left(\tau_k - \dfrac{x}{a}\right) - \left(\tau_{k - 1} - \dfrac{x}{a}\right) \,\right] = \\[3mm]
	=\; & \dfrac{1}{a} \left(\, \alpha_m \left[\, \gamma(0) -
	\gamma\!\left(\tau_m - \dfrac{x}{a}\right) \,\right] + \sum\limits_{k = m + 1}^{p} \alpha_k \left[\,\gamma\left(\tau_{k - 1} - \dfrac{x}{a}\right) - \gamma\left(\tau_k - \dfrac{x}{a}\right) \,\right] \,\right).
\end{aligned}
\end{equation*}

Пусть $x \ge aT$. Согласно \eqref{int_eta_w_def} получим
\begin{equation*}
	\dfrac{d}{dx}\int\limits_0^T \eta(t)w(x,t) \,dt = 
	\dfrac{d}{dx} \left[\, \gamma(0)\sum\limits_{k = 1}^{p} \alpha_k(\tau_k - \tau_{k - 1}) \,\right] = 0.
\end{equation*}

В результате имеем
\begin{equation} \label{d_int_eta_w_def}
\dfrac{d}{dx}\int\limits_0^T \eta(t)w(x,t) \,dt = 
	\begin{cases}
		\, \dfrac{1}{a} \,\sum\limits_{k = m + 1}^{p} \alpha_k \left[\,\gamma\left(\tau_{k - 1} - \dfrac{x}{a}\right) - \gamma\left(\tau_k - \dfrac{x}{a}\right) \,\right] + \\[7mm]
		\, +\;\dfrac{1}{a}\,\alpha_m \left[\, \gamma(0) -\gamma\!\left(\tau_m - \dfrac{x}{a}\right) \,\right], \qquad
		a\tau_{m - 1} \le x \le a\tau_m, \\[5mm]
		\, 0, \qquad x > aT.
	\end{cases}
\end{equation}

Для оценки точности вычисления функции $\varphi(x)$ выводится график разности
\begin{equation} \label{E}
	\int\limits_{0}^{T} \eta(t) u(x,t)\,dt - \widetilde{\psi}(x).
\end{equation}
Соответственно, необходимо вычислить интеграл
\begin{equation} \label{int_eta_u}
	\int\limits_{0}^{T} \eta(t) u(x,t)\,dt.
\end{equation}

По определению решения нелокальной задачи
\begin{equation*}
u(x,t) = U(t)\varphi(x) = 
	\begin{cases}
	\, \varphi(x - at), \qquad & x > at, \\
	\, 0, & x \le at.
	\end{cases}
\end{equation*}
Следовательно, будем рассматривать интеграл \eqref{int_eta_u} в зависимости от соотношения $x - at$.

Пусть $x < aT$. Положим $a\tau_{m - 1} \le x \le a\tau_m$. Имеем
\begin{gather*}
	\int\limits_{0}^{T} \eta(t) u(x,t)\,dt = \sum\limits_{k = 1}^{m - 1} \alpha_k \int\limits_{\tau_{k - 1}}^{\tau_k}u(x,t) \,dt +
	\alpha_m \int\limits_{\tau_{m - 1}}^{x/a}u(x,t) \,dt = \\[2mm]
	= \sum\limits_{k = 1}^{m - 1} \alpha_k \int\limits_{\tau_{k - 1}}^{\tau_k}\varphi(x - at) \,dt +
	\alpha_m \int\limits_{\tau_{m - 1}}^{x/a}\varphi(x - at) \,dt.
\end{gather*}

Пусть $x \ge aT$. Тогда
\begin{equation*}
	\int\limits_{0}^{T} \eta(t) u(x,t)\,dt = \sum\limits_{k = 1}^{p} \alpha_k \int\limits_{\tau_{k - 1}}^{\tau_k}u(x,t) \,dt =
	\sum\limits_{k = 1}^{p} \alpha_k \int\limits_{\tau_{k - 1}}^{\tau_k}\varphi(x - at) \,dt.
\end{equation*}

Теперь посчитаем интеграл
\begin{equation} \label{int_phi}
	\varPhi(x, t_1, t_2) = \int\limits_{t_1}^{t_2}\varphi(x - at) \,dt.
\end{equation}
Имеем
\begin{equation*}
\begin{gathered}
	\int\limits_{t_1}^{t_2}\varphi(x - at) \,dt = a\,\sum\limits_{n = 0}^{n_0 - 1} \frac{1}{\alpha_1^{\,\,n + 1}}
	\sum\limits_{\substack{{k_i \ge 0} \\_{k_1 + ... + k_p = n}}}\!\!\!\!C_n^{\,k_1, ..., k_p}\;
	\prod_{m = 1}^{p}(\alpha_m - \alpha_{m + 1})^{k_m} \,\times \\[5mm]
	\times\, \int\limits_{t_1}^{t_2} \Theta\left(x - at - a\sum\limits_{i = 1}^{p}k_i\tau_i\right)
	\psi\,'\left(x - at - a\sum\limits_{i = 1}^{p}k_i\tau_i\right) dt.
\end{gathered}
\end{equation*}
Отдельно посчитаем интеграл в правой части равенства. 
\begin{gather*}
	\int\limits_{t_1}^{t_2} \Theta\left(x - at - a\sum\limits_{i = 1}^{p}k_i\tau_i\right)
	\psi\,'\left(x - at - a\sum\limits_{i = 1}^{p}k_i\tau_i\right) dt = \\[5mm]
	= \dfrac{1}{a} \int\limits_{x - a\left(t_2 + \sum\limits_{i = 1}^{p}k_i\tau_i\right)}
	^{x - a\left(t_1 + \sum\limits_{i = 1}^{p}k_i\tau_i\right)} \Theta(t)\psi\,'(t) \,dt = \\[5mm]
	= \dfrac{1}{a} \left\{\; \Theta\left(x - a\left[t_1 + \sum\limits_{i = 1}^{p}k_i\tau_i\right]\right)
	\psi\left(x - a\left[t_1 + \sum\limits_{i = 1}^{p}k_i\tau_i\right]\right) - \right. \\[5mm] \left.
	-\, \Theta\left(x - a\left[t_2 + \sum\limits_{i = 1}^{p}k_i\tau_i\right]\right)
	\psi\left(x - a\left[t_2 + \sum\limits_{i = 1}^{p}k_i\tau_i\right]\right) \;\right\}.
\end{gather*}


Таким образом,
\begin{equation} \label{int_phi_def}
\begin{gathered} 
	\varPhi(x, t_1, t_2) = a\,\sum\limits_{n = 0}^{n_0 - 1} \frac{1}{\alpha_1^{\,\,n + 1}}
	\sum\limits_{\substack{{k_i \ge 0} \\_{k_1 + ... + k_p = n}}}\!\!\!\!C_n^{\,k_1, ..., k_p}\;
	\prod_{m = 1}^{p}(\alpha_m - \alpha_{m + 1})^{k_m} \times \\[5mm] \times\,
	\dfrac{1}{a} \left\{\; \Theta\left(x - a\left[t_1 + \sum\limits_{i = 1}^{p}k_i\tau_i\right]\right)
	\psi\left(x - a\left[t_1 + \sum\limits_{i = 1}^{p}k_i\tau_i\right]\right) - \right. \\[5mm] \left. -\, 
	\Theta\left(x - a\left[t_2 + \sum\limits_{i = 1}^{p}k_i\tau_i\right]\right)
	\psi\left(x - a\left[t_2 + \sum\limits_{i = 1}^{p}k_i\tau_i\right]\right) \;\right\}.
\end{gathered}
\end{equation}

Окончательно, имеем
\begin{equation} \label{int_eta_u_def}
\int\limits_{0}^{T} \eta(t) u(x,t)\,dt =
	\begin{cases}
	\, \sum\limits_{k = 1}^{m - 1} \alpha_k \varPhi(x, \tau_{k - 1}, \tau_k) + \alpha_m \varPhi(x, \tau_{m - 1}, x/a), 
	\quad \dfrac{x}{a} \in [\tau_{m - 1}, \tau_m], \\[5mm]
	\, \sum\limits_{k = 1}^{p} \alpha_k \,\varPhi(x, \tau_{k - 1}, \tau_k), \quad \quad \dfrac{x}{a} > T.
	\end{cases}
\end{equation}

Программа написана на языке Python.

Ввод данных осуществляется через диалоговое окно. На вход подаются следующие параметры.
\begin{enumerate}
	\item $l$~--- положительное число, длина рассматриваемого отрезка $[0,l]$.
	\item $T$~--- положительное число, время наблюдения за процессом.
	\item $a$~--- положительное число, скорость переноса вещества.
	\item $\tau$~--- массив вводимых положительных чисел $\tau_1, ..., \tau_{p - 1}$, упорядоченных по возрастанию; 
	внутренние точки разбиения отрезка $[0,T]$.
	\item $\alpha$~--- массив чисел $\alpha_1, ..., \alpha_p$, последовательные дискретные
	значения кусочно постоянной весовой функции $\eta(t)$, заданные по следующиму принципу
	\begin{equation*}
		\eta(0) = \alpha_1, \; \eta(T) = \alpha_p; \qquad \eta(t) = \alpha_i, \quad t \in (\tau_{i - 1}, \tau_i), \quad i = 1, ..., p.
	\end{equation*}  
	\item $\gamma(t)$~--- действительная функция, соответствующая граничному значению $u(0,t) = \gamma(t)$, $ \quad 0 \le t \le T$.
	\item $\psi(x)$~--- действительная функция, соответствующая нелокальному по времени $t$ условию
	\begin{equation*}
		\int\limits_{0}^{T} \eta(t) u(x, t)\,dt = \psi(x), 	\qquad 0 \le x \le l.
	\end{equation*}
\end{enumerate}
Дополнительно предусмотрена возможность ввода натурального числа $N$~--- количества на единичном отрезке. 
Это число используется для определения числа $|\, \omega \,|$~--- количества точек сетки $\omega = \omega[0,l]$ 
по формуле $|\, \omega \,| = \lceil\, N \!\cdot\! l \,\rceil$.

Напомним, что сеткой $\omega[0,l]$ называется совокупность точек $x_0$, $x_1$, ..., $x_n$ из отрезка $[0,l]$ таких, что
\begin{equation*}
	0 = x_0 < x_1 < ... < x_{n - 1} < x_n = l.
\end{equation*}

Сетка $\omega[0,l]$ выбрана равномерной. В точках сетки $\omega[0,l]$ вычисляется искомая функция $\varphi(x)$ и 
по полученному массиву значений строится график функции $\varphi(x)$.

В программе реализована валидация (проверка на корректность) входных данных. Все поля ввода снабжены значениями по умолчанию.

На выходе получаем график функции $\varphi(x) = u(x,0)$~--- начального состояния системы.

Также предусмотрена возможность построения решения $u(x,t)$ в последовательные моменты времени $t \ge 0$.

\subsection{Алгоритм}
\begin{enumerate}
	\item Определяются константы $l$, $T$, $a$, массивы $\tau = [\tau_1, \tau_2, ..., \tau_p, 0]$, 
	$\alpha = \linebreak = [\alpha_1, \alpha_2, ..., \alpha_p, 0]$. Нумерация в массивах начинается с единицы. $p + 1$~-ый элемент 
	положен нулевым для удобства вычислений. 
	\item Определяются функции $\gamma(t)$, $\psi(x)$.
	\item Вычисляется константа
	\begin{equation*}
		\int\limits_0^T \eta(t)\gamma(t) \,dt.
	\end{equation*}
	\item Определяются функции 
	\begin{equation*}
		\int\limits_0^T \eta(t)w(x,t) \,dt, \quad \dfrac{d}{dx}\int\limits_0^T \eta(t)w(x,t) \,dt
	\end{equation*}
	по формулам \eqref{int_eta_w_def}, \eqref{d_int_eta_w_def} соответсвенно.
	\item Определяются функции $\widetilde{\psi}(x)$, $\widetilde{\psi}\,'(x)$ 
	по формулам \eqref{tilde_psi}, \eqref{d_tilde_psi} соответсвенно.
	\item Определяется функция $\varphi(x) = u(x,0)$. 
	\item Строится сетка $\omega[0,l]$. 
	\item В узлах сетки вычисляется функция $\varphi(x)$ и по получаемому массиву значений 
	строится график этой функции.
	\item В узлах сетки вычисляются функции 
	\begin{equation*}
		\int\limits_0^T \eta(t)u(x,t) \,dt \quad\text{и}\quad \widetilde{\psi}(x)
	\end{equation*} 
	и по получаемым массивам значений строятся их графики.
\end{enumerate}

Отдельно рассмотрим функцию $\varphi(x)$. Она имеет следующий вид 
\begin{equation*}
\begin{aligned}
	& \varphi(x) = a\,\sum\limits_{n = 0}^{n_0 - 1} \frac{1}{\alpha_1^{\,\,n + 1}}
	\sum\limits_{\substack{{k_i \ge 0} \\_{k_1 + ... + k_p = n}}}\!\!\!\!C_n^{\,k_1, ..., k_p}\;
	\prod_{m = 1}^{p}(\alpha_m - \alpha_{m + 1})^{k_m} \,\times \\[2mm] \times\,
	& \Theta\left(x - a\sum\limits_{i = 1}^{p}k_i\tau_i \right)\psi\,'\left(x - a\sum\limits_{i = 1}^{p}k_i\tau_i\right), 
	\qquad 0 \le x \le l.
\end{aligned}
\end{equation*}
Ниже приведена её программная реализация в виде псевдокода.
\begin{lstlisting}[language=C]
phi(x) {
	res = 0    // возвращаемое значение
	n_0 = ceil(x / a / tau[1])
	b = 1 / alpha[1]
	for (n = 0; n < n_0; n++){
		Sum = 0    // внешняя сумма
		for (array i: i[j] >= 0, j = 1, ..., p, i[1] + ... + i[p] = n){    
		// пройтись по мультиномиальным индексам
			sum = 0    // внутренняя сумма
			for (q = 1; q < p + 1; q++){
				sum *= i[q] * tau[q]
			if (x - a*sum < 0) // в силу нильпотентности полугруппы
				continue 
			}
			prod = 1
			for (q = 1; q < p + 1; q++){
				prod *= (alpha[q] - alpha[q + 1])^(i[q])    
				// ^ - возведение в степень; alpha[p + 1] = 0
			}
			Sum += d_tilde_psi(x - a*sum) * prod    
			// d_tilde_psi(x) - производная переопределённой функции tilde_psi(x)
		} 
		res += b * sum
		b *= 1 / alpha[1]
	}
	res *= a
	return res
}

\end{lstlisting}

\newpage

\section{Результаты вычислений}

Входные данные:
$l = 10$,  $T = 5$,  $a = 1$, $\tau_1 = 2.5$, $[\alpha_1;\, \alpha_2] = [2;\, 1]$, 
$\gamma(t) \equiv 0$, $\,\psi(x) = x\exp(-x)$.
\begin{figure}[H]
	\centering
	\includegraphics[trim={2.4cm, 0.6cm, 0, 1.5cm}, clip, scale=0.62]{First1.png}
	\caption{Заданная функция $\psi(x) = x\exp(-x)$}
	\label{fig:image1}
\end{figure}

\begin{figure}[H]
	\centering
	\includegraphics[trim={2.1cm, 0.6cm, 0, 1.5cm}, clip, scale=0.615]{First2.png}
	\caption{Найденная функция $\varphi(x)$}
	\label{fig:image2}
\end{figure}

Начальное условие $\varphi(x)$ соответствует обобщённому решению $u(x,t)$ в пространстве $L_1[0,l]$.
\goodbreak
\newpage	

Входные данные: 
$l = 10$,  $T = 5$,  $a = 1$, $\tau_1 = 2.5$, $[\alpha_1;\, \alpha_2] = [1;\, 2]$, 
$\gamma(t) \equiv 0$, $\,\psi(x) = x \sin(x/4)$.
\begin{figure}[H]
	\centering
	\includegraphics[trim={2.5cm, 0, 0, 1.5cm}, clip, scale=0.62]{Second1.png}
	\caption{Заданная функция $\psi(x) = x \sin(x/4)$}
	\label{fig:image3}
\end{figure}

\begin{figure}[H]
	\centering
	\includegraphics[trim={2.5cm, 0, 0, 1.5cm}, clip, scale=0.62]{Second2.png}
	\caption{Найденная функция $\varphi(x)$}
	\label{fig:image4}
\end{figure}

Начальное условие $\varphi(x)$ соответствует обобщённому решению $u(x,t)$ в пространстве $L_1[0,l]$. При этом
$\varphi(x) \in C[0,l]$.

\newpage	

Входные данные: 
$l = 10$,  $T = 5$,  $a = 1$, $\tau_1 = 2.5$, $[\alpha_1;\, \alpha_2] = [1;\, 2]$, 
$\gamma(t) \equiv 0$, $\,\psi(x) = x^3(10 - x)$.
\begin{figure}[H]
	\centering
	\includegraphics[trim={2.1cm, 0, 0, 1.5cm}, clip, scale=0.61]{Third1.png}
	\caption{Заданная функция $\psi(x) = x^3(10 - x)$}
	\label{fig:image5}
\end{figure}

\begin{figure}[H]
	\centering
	\includegraphics[trim={2.02cm, 0, 0, 1.5cm}, clip, scale=0.61]{Third2.png}
	\caption{Найденная функция $\varphi(x)$}
	\label{fig:image6}
\end{figure}

Начальное условие соответсвует классическому решению $u(x,t)$ в пространстве $L_1[0,l]$. При этом
$\varphi(x) \in C^2[0,l]$.

\newpage	

%Входные данные: 
%$l = 10$,  $T = 5$,  $a = 1$, $[\tau_1;\, \tau_2] = [1.666;\, 3.333]$, \linebreak 
%$[\alpha_1;\, \alpha_2;\, \alpha_3] = [1;\, 3;\, 5]$, $\gamma(t) \equiv 0$, $\,\psi(x) = x - \sin(x)$.
%\begin{figure}[H]
%	\centering
%	\includegraphics[trim={2.475cm, 0, 0, 1.5cm}, clip, scale=0.624]{Forth1.png}
%	\caption{Заданная функция $\psi(x) = x - \sin(x)$}
%	\label{fig:image7}
%\end{figure}
%
%\begin{figure}[H]
%	\centering
%	\includegraphics[trim={2.3cm, 0, 0, 1.5cm}, clip, scale=0.62]{Forth2.png}
%	\caption{Найденная функция $\varphi(x)$}
%	\label{fig:image8}
%\end{figure}
%
%\newpage	
%
%Входные данные: 
%$l = 10$,  $T = 5$,  $a = 1$, $[\tau_1;\, \tau_2] = [1.666;\, 3.333]$, \linebreak 
%$[\alpha_1;\, \alpha_2;\, \alpha_3] = [1;\, 2.5;\, 1]$, $\gamma(t) \equiv 0$, $\,\psi(x) = \sin(x) - \arctg(x)$.
%\begin{figure}[H]
%	\centering
%	\includegraphics[trim={2.15cm, 0, 0, 1.5cm}, clip, scale=0.617]{Fifth1.png}
%	\caption{Заданная функция $\psi(x) = \sin(x) - \arctg(x) $}
%	\label{fig:image9}
%\end{figure}
%
%\begin{figure}[H]
%	\centering
%	\includegraphics[trim={2.3cm, 0, 0, 1.5cm}, clip, scale=0.62]{Fifth2.png}
%	\caption{Найденная функция $\varphi(x)$}
%	\label{fig:image10}
%\end{figure}
%
%\newpage

\section*{Заключение}
\addcontentsline{toc}{section}{Заключение}
В результатом работы получено решение нелокальной задачи для уравнения простого переноса с поглощением и реализована программа,
строящяя это решение при $\sigma(x) \equiv 0$, $\gamma(t) \equiv 0$. 

Решение исходной нелокальной задачи получается как решение нелокальной задачи для дифференциального уравнения в банаховом пространстве 
при выборе конкретных пространства $E$ и оператора $A$. Сама абстрактная задача решается методом полугрупп, причём в рассматриваемом 
случае полугруппа нильпотентна. Решение представляется явной формулой в виде конечной суммы. 

Что касается программы, кратко и ясно приведено её описание. Она в точности реализует формулу решения нелокальной задачи. Полученные графики начальных данных и соответствующих им решений продемонстрированы чётко и наглядно.

\newpage

\begin{thebibliography}{99}
	
\addcontentsline{toc}{section}{\bibname}

	\bibitem{Tikhonov1} Тихонов~И.~В., Ву~Нгуен~Шон~Тунг.
	\emph{Разрешимость нелокальной задачи для эволюционного уравнения переноса с суперустойчивой полугруппой}
	// Дифференциальные уравнения (в печати).
	
	\bibitem{Tikhonov2} Тихонов~И.~В., Ву~Нгуен~Шон~Тунг.
	\emph{Формулы явного решения в модельной нелокальной задаче для уравнения простого переноса}
	// Математические заметки СВФУ. 2017. Т. 24, № 1 (93). С. 57-73.
	
	\bibitem{Balakrishnan_1} Balakrishnan~A.~V.
	\emph{On superstability of semigroups}
	// In: M.P. Polis et al (eds.). Systems modelling and optimization. 
	Proceedings of the 18th IFIP conference on system modelling and optimization. 
	CRC research notes in mathematics. Chapman and Hall. 1999. P. 12–19.
	
	\bibitem{Balakrishnan_2} Balakrishnan~A.~V.
	\emph{Smart structures and super stability} 
	// In: G. Lumer, L. Weis (eds.). Evolution
	equations and their applications in physical and life sciences. Lecture notes in pure and applied
	mathematics. Marcel Dekker. 2001. V. 215. P. 43–53.
	
	\bibitem{Dunford_Schwartz} Данфорд~Н., Шварц~Д.
	\emph{Линейные операторы.} М., 1962, Т.1. Общая теория.

	\bibitem{Pazy} Pazy~A.
	\emph{Semigroups of linear operators and applications to partial differential equations.} N.Y.: Springer Verlag, 1983.
	
	\bibitem{Kolmogorov_Fomin} Колмогоров~А.~Н., Фомин~С.~В.
	\emph{Элементы теории функций и функционального анализа.} М.: Наука, 1976.
	
	\bibitem{Trenogin} Треногин~В.~А.
	\emph{Функциональный анализ.} 4-е изд. М.: ФИЗМАТЛИТ, 2007.
	
	\bibitem{Filippov} Филиппов~А.~Ф.
	\emph{Введение в теорию дифференциальных уравнений.} М.: Едиториал УРСС, 2004.
\end{thebibliography}

\end{document}
